\documentclass[b5paper]{report}
\usepackage[lmargin=25mm,rmargin=25mm,tmargin=27mm,bmargin=30mm]{geometry}

\usepackage[toc]{appendix}
\usepackage{graphicx}
\usepackage{subcaption}
\usepackage{listings}
\usepackage{color}
\usepackage{xcolor}
\usepackage{tikz}
\usepackage{hyperref}
\usepackage{parskip}
\usetikzlibrary{positioning,shapes,decorations,calc,fit,arrows.meta}

\usepackage{libertine}
\usepackage{inconsolata}
\usepackage{libertinust1math}
\usepackage[T1]{fontenc}
\catcode`\_=12

\begin{document}
\title{Implementing Concurrent Memory\linebreak Reclamation Schemes}
\author{Martin Hafskjold Thoresen}
\date{\today}
\maketitle

\newcommand{\note}[1]{{\color{olive}\sffamily #1}}
\newcommand{\todo}[1]{{\color{purple}\sffamily[TODO: #1]}}
\newcommand{\code}[1]{{\ttfamily #1}}

\newcommand{\rustc}{\code{rustc}}
\newcommand{\cargo}{\code{cargo}}
\newcommand{\rustup}{\code{rustup}}



\lstset{ %
 backgroundcolor=\color{white}, % choose the background color; you must add \usepackage{color} or \usepackage{xcolor}; should come as last argument
 basicstyle=\footnotesize\ttfamily,             % the size of the fonts that are used for the code
% breakatwhitespace=false,      % sets if automatic breaks should only happen at whitespace
 breaklines=true,              % sets automatic line breaking
% captionpos=b,                 % sets the caption-position to bottom
 commentstyle=\color{gray},    % comment style
% deletekeywords={...},         % if you want to delete keywords from the given language
% escapeinside={\%*}{*)},       % if you want to add LaTeX within your code
% extendedchars=true,           % lets you use non-ASCII characters; for 8-bits encodings only, does not work with UTF-8
% frame=single,	                % adds a frame around the code
% keepspaces=true,              % keeps spaces in text, useful for keeping indentation of code (possibly needs columns=flexible)
 keywordstyle=\color{purple},    % keyword style
 language=C,                    % the language of the code
 keywords={let,mut,fn,return,pub,struct,where,impl},
% morekeywords={*,...},         % if you want to add more keywords to the set
 numbers=left,                  % where to put the line-numbers; possible values are (none, left, right)
 numbersep=5pt,                 % how far the line-numbers are from the code
 numberstyle=\color{gray}\ttfamily,           % the style that is used for the line-numbers
% rulecolor=\color{black},      % if not set, the frame-color may be changed on line-breaks within not-black text (e.g. comments (green here))
 showspaces=false,              % show spaces everywhere adding particular underscores; it overrides 'showstringspaces'
 showstringspaces=false,        % underline spaces within strings only
% showtabs=false,               % show tabs within strings adding particular underscores
% stepnumber=2,                 % the step between two line-numbers. If it's 1, each line will be numbered
% stringstyle=\color{mymauve},  % string literal style
% tabsize=2,	                  % sets default tabsize to 2 spaces
% title=\lstname                % show the filename of files included with \lstinputlisting; also try caption instead of title
 frame=single,
 rulecolor=\color{lightgray}
}


\begin{abstract}
  In systems without a garbage collector dynamic allocated memory have to freed.
  In concurrent systems it is a hard problem to find out when shared memory is
  safe to free. In this report we look at two of the most used schemes for
  concurrent memory reclamation, \emph{Epoch Based Reclamation} and \emph{Hazard
  Pointers}, including implementations in the Rust programming language and
  experimental results of their performance overhead.
\end{abstract}

\tableofcontents

\chapter{Introduction}

This report is the final report of the course TDT4501: Computer Science
Specialization Project, worth 15 SP (credits) at the Norwegian University of
Science and Technology in the fall of 2017

\section*{}

In 1965 Gordon Moore stated what is now knows as \emph{Moore's Law}, which
roughly states that the number of transistors on integrated circuits will double
every 18 months. The law has also been used to describe the increase in clock
speed of central processing units (CPU's). During the 2000s however, this growth
slowed drastically, and CPU's sold today have very similar clock speeds as those
sold in 2005, over 10 years ago. Despite the clock speed stagnation, there have
been major performance improvements in the last 15 years. We have more cache,
improved branch predictors, larger vector instructions, super-scalar
architectures, and \emph{multi-core systems}, the latter of which is central to
this report. Multi-core systems are now ubiquitous; high end desktop CPU's with
16 cores are available, and even smartphones with 8 cores are on the market.

The introduction and growth of multi-core systems, however, has not been as
noticeable as the former improvements in clock speed. One reason for this is
that many real world problems and systems are inherently serial problems. Due to
Amdahls Law, this even puts an upper bound on the possible speedup we can get by
using concurrency.  Another reason is that programs already written cannot
automatically use the extra parallelism that newer architectures give us;
parallelism needs to be accounted for by programmers, and often imply a major
shift in program architecture. A third reason may be that concurrent systems are
extremely complex. Bugs are (usually) not reproducible, tests are not
deterministic, and the combinatorial explosion of possible interwindings of
executions across threads makes a brute force approach to program verification
intractable.

In this report we look at the problem of concurrent memory reclamation. The
problem can be informally stated as ``how to we know when it is safe to free
allocated memory in a concurrent system?''. In Chapter~\ref{ch:background} we
look at the needed prerequisites in order to fully understand the problem and
its nuances; in Chapter~\ref{ch:memory-reclamation} we look at different memory
reclamation schemes, both the ones implemented here and a highlight of newer
variants; in Chapter~\ref{ch:methodology} we look at the practical side of this
project: implementing memory reclamation schemes in Rust, as well as testing and
benchmarking methods; in Chapter~\ref{ch:results} we look at the experimental
results from running the system implemented; and in Chapter~\ref{ch:conclusion}
we summarize the project, comment on methodologies, and suggest future work.
Appendix~\ref{ch:rust-toolchain} contains practicalities around the Rust
ecosystem, and Appendix~\ref{ch:benchmarks} contains complete benchmark data for
the experimental results obtained.




\chapter{Background\label{ch:background}}

The field of computer science is broad and diverse, so readers may have very
different prerequisites.  We will assume the reader is familiar with traditional
memory management (as in C), and the basics of computer hardware, such as the
existence of cache memory. The report will (hopefully) be approachable to anyone
with this background.

\section{Terminology} We start with basic terminology of memory management. Most
of the programs we write \emph{allocate} memory, meaning requesting a memory
range from the OS for exclusive use by the program. This memory is then
\emph{de-allocated}, or \emph{freed}, when we are done with it, so that the
memory may be used at a later time for a different purpose; the memory is thus
\emph{reclaimed}. If we want to ``mark'' memory as freed without freeing it
(that is, the program still has exclusive use of it) we say we \emph{retire} the
memory (reasons for doing so will become apparent). In concurrent settings, we
risk accessing memory at the same time; a \emph{data race}, or a \emph{race
condition} is when multiple threads are operating on the same memory
simultaneously, and where at least one operation is a write.

Both modern compilers and CPU's may reorder program instructions if they think it
will improve performance, for instance by improving memory locality and thus
improving cache behavior, or reduce pipeline stalling. This may or may not be
acceptable by the program. The guarantee that the instructions executed appear
to take effect in program order is called \emph{sequential consistency} (SC).
We call a system in which threads can be prevented from making progress
\emph{blocking}. The opposite of blocking is \emph{non-blocking}. A system is
\emph{lock-free} if we have a guarantee that \emph{some} thread will make
progress. Note that any given thread may be blocked for an infinite amount of
time. The guarantee that \emph{any} given thread will eventually make progress
is called \emph{wait-freedom}.  For a more thorough introduction
see~\cite{herlihy2011art}.

In communities dealing with concurrency and parallelism the meaning of the two
words is often debated. We will settle for the informal definitions given
in~\cite{pacheco2011introduction}. An informal distinction is that in a parallel
system multiple tasks run simultaneously, while in a concurrent system multiple
tasks may be in progress at the same time.


\section{Rust}\label{sec:rust}

Rust~\cite{rust} is a programming language which focus is safety, performance,
and concurrency. It is is freely developed by over 1900 contributors on the
version control platform GitHub~\cite{github}, and is officially sponsored by
Mozilla. Rust 1.0 was released in May 2015, and the current stable version of
the language is 1.22, with a new version released every 6th week. The language
is compiled and typed, and includes features such as type inference, pattern
matching, tagged enums, template-like generics, and a minimal runtime without a
garbage collector. As of November 2017 the official Rust compiler \code{rustc}
uses LLVM~\cite{llvm} for optimization and code generation. The language has no
formal specification, and language changes are done through an RFC (``request
for comments'') process. The syntax of Rust is out of scope for this report, so
we will settle with a highlight of the most important features of Rust. For a
thorough introduction to Rust, see The Rust Programming Language\cite{trpl}.

As C++, Rust has structs. It does not however have classes or inheritance. The
type system is based around \emph{traits}, which are comparable to interfaces in
Java. A trait defines methods, optionally with an implementation. We can then
\emph{implement} a trait for a struct. Implementing a trait for a struct
requires that either the trait of the struct is defined in the \emph{crate} (the
project unit) the implementation is in. A consequence of this is that it is
possible to extend types from the standard library with ones own traits. It is
possible to have values which type is a trait, called \emph{trait objects}. This
is useful when we want to be agnostic of the implementation of an interface.
However, this requires dynamic dispatch of the function calls of the value,
which imposes a runtime overhead.

Despite the lack of a garbage collector, Rust programmers does not have to
manually manage memory. Rust solves this through \emph{ownership} semantics:
values are either \emph{owned} or \emph{borrowed} by its scope. Transferring
ownership of a value is possible by \emph{moving} it. When a borrowed value goes
out of scope, nothing happens as a borrowed value is simply a pointer. When an
owned value goes out of scope it is \emph{dropped}, meaning its destructor is
ran. Rust's ownership rules, which are enforced by the compiler, guarantees that
there are no other references to the object when it is dropped. Implementing the
destructor of a type is done by implementing the \code{Drop} trait. There is no
trait for constructors, but by convention a method called \code{new} is used.

An example of an abstraction using both \code{new} and \code{Drop} is
\code{std::box::Box<T>}, a fat pointer to a heap allocated value of type
\code{T}. The \code{new} implementation allocates the memory, and \code{drop}
will free the memory. This pattern is often called \emph{RAII} (resource
allocation is initialization). This way we get a safe abstraction over heap
allocation, since the memory is freed exactly once when the \code{Box} goes out
of scope.  \code{Box<T>} also implements another trait, \code{Deref<T>}, which
causes \code{Box} references to automatically be converted by the compiler to
\code{T} references, if needed. We note that we can \emph{not} pass a
\code{Box<T>} to a function that takes a \code{T}, since the two types have
different \code{drop} implementations.

In addition to the ownership rules, Rust have two important rules for
references: references are always valid, and mutable references are exclusive.
At compile time Rust tracks the \emph{lifetime} of all values, and ensures that
the lifetime of all references does not outlive the lifetime of the data they
are referencing. If a function returns a reference to a value allocated on the
stack of that function it is caught at compile time, since the reference's
lifetime would outlive the functions lifetime, and the values lifetime is at
most that of the function. The second rule stops one reference to invalidate the
data that another reference points to. For instance it is not allowed (and
caught at compile time) to have a reference to a value inside a \code{Vec} and
mutate the \code{Vec}, since mutating the \code{Vec} might cause the data to be
moved, which invalidates the reference.

\subsection{Unsafe Rust}

Sometimes the rules of Rust are too strict. Rust, aiming to be a pragmatic
language, offers an escape hatch for this: \emph{unsafe code}. In unsafe code it
is allowed to: dereference raw pointers, call \code{unsafe} functions, implement
\code{unsafe} traits, and mutate static variables. Raw pointers are, as in C and
C++, just a memory address, and as with references, both a immutable
(\code{*const T}) and mutable (\code{*mut T}) variants exist. The difference
between references and raw pointers is that raw pointers do not have lifetimes
attached to them. In unsafe Rust, we enter the dangerous and error-prone
territory that C and C++ programmers know all too well. Examples of
\code{unsafe} functions are all FFI functions, unchecked array indexing, and
creating a \code{String} without UTF-8 checks (all \code{String}s in Rust are
valid UTF-8). \code{unsafe} code is often more about signaling to the programmer
that the code is fragile, and that one should put extra care into making sure
that ones assumptions are correct. For instance, dereferencing raw pointers is
\code{unsafe}, but it may happen that we know that the pointer is still valid,
eg.\ if it points to heap memory that we explicitly leaked, using
\code{std::mem::forget}.

By exposing \code{unsafe} to programmers, one might argue that Rust is just as
unsafe as C or C++. The primary advantage of \code{unsafe} is that the ratio of
unsafe to safe code is very low. For instance, the Rust compiler itself consists
only of $1\%$ unsafe code\cite{rustc-unsafe}. This makes it possible to focus on
the few places in a code base that is unsafe when developing, testing, and code
reviewing. It is however important to realize that the effects of unsafe code is
non-local: bugs in unsafe code can cause undefined behavior to take effect in
perfectly safe code.



\section{Concurrency in Hardware}

Most programmers rarely have to think about the hardware their programs run on.
Caches, for instance, is designed to be transparent for programmers. With
concurrency this changes, as concurrent programming have challenges in which a
mental model of how the hardware works is paramount. In this section we consider
multi-core CPU's with multiple levels of cache; this includes most modern CPU's,
both desktop computers and laptops, as well as most mobile devices, such as
smartphones and tablets.

The cache hierarchy poses a problem for shared memory programs. Each core
usually have their own L1 cache, while the remaining levels, L2 (and maybe L3),
are shared between cores. This means that when a program is writing to memory,
the same logical memory may be present in multiple physical locations on the
CPU.\@ This causes synchronization problems when this memory is written to. This
is the problem of \emph{cache coherence}. A memory range does not even have to
be used by multiple threads: since caches operate on memory chunks, called
\emph{cache lines}, it is enough that two values are on the same cache line for
synchronization problems to occur. This is called \emph{false sharing}.

We would like our programs to be sequentially consistent, as this maps the
program execution to the source code of the program. However, we also want our
programs to be performant, and it turns out that from a hardware perspective
these two requirements are conflicting. CPU's are simply not sequentially
consistent.

CPU architectures have rules on how much the CPU is allowed to reorder reads and
writes without breaking the semantics of the program when ran sequentially. They
also have instructions explicitly for avoiding instruction reordering, called
\emph{memory fences} (or \emph{barriers}; we will use the former name). The x86
architecture have a rather strong \emph{memory model}, meaning the CPU is very
limited in its ability to reorder.  For instance, x86 forbids reads to be
reordered with other reads, and writes with other writes. It also forbids any
reordering with locked instructions; this is very strict, as the preferred
instruction used for sequential consistent atomic store is a locked instruction,
meaning atomic stores are full fences. ARM on the other hand, have a comparably
weak memory model. A disadvantage of this is that sequentially consistent
atomics require full fences for both loads and stores, since they are allowed to
be reordered. ARMv8 offers a solution to this by having explicit instructions
for SC loads and stores. For more information on x86, see Chapter 8.2.2
in~\cite{intel64}; for information about ARMv7 see~\cite{armv7-reference-manual},
and for ARMv8 see~\cite{armv8-reference-manual}.

Many instruction sets have \emph{atomic} operations. An operation is atomic if
its is performed either fully or not at all. For instance, if thread A
non-atomically stores a value at a location, thread B may observe the write when
it is only partially performed. This is not possible using atomic instructions.
Atomic load and store is the two most fundamental atomic operations, but most
platforms also give us additional instructions; the most notable being
\code{compare-and-swap} (\code{CAS}) and \code{fetch-and-add} (\code{FAA}).
\code{CAS} takes three arguments: a memory address, a value A, and a value B. If
the value at the given address is A, it writes B to the address. If not, nothing
is done. \code{FAA} atomically increments a memory location by the specified
value. \code{CAS} is a central building block in concurrent programming, and
especially in lock-free programming.

\section{Concurrency in Software}

Modern operating systems run programs as \emph{processes}. Processes may spawn
other processes or \emph{threads} for concurrent program execution. Processes
and threads differ in that threads share the memory space of their parent
process, while processes do not. Since we are looking at memory reclamation, we
will only be looking at shared memory concurrency, meaning concurrency using
threads.

Synchronization between threads is usually done through either sharing memory or
message passing. When sharing memory we have constructs like \emph{mutexes},
\emph{semaphores}, which protects critical sections, and \emph{condvars} which
allow threads to wait until a particular condition is met. Message passing is a
communication technique in which threads communicate over \emph{channels}.
The channels may have single- or multiple senders and/or receivers. Message
passing (MP) has seen an increased popularity with the programming language Go, in
which MP is the preferred way of thread communication.

Thread management is also topic within the realm of software concurrency.
Threads spawned by the operating system gives a \emph{handle} to the spawning
thread, and the spawning thread may block its execution until the thread is done,
by \emph{joining} it. Thread management involves the structure of thread
handles, when to spawn and when to join, as well as simple communication
between the threads. Popular examples of high level thread management schemes
involved thread pools, green threads (lightweight threads), and work stealing.
Notable examples of projects employing such schemes are Cilk~\cite{cilk} and
its Rust counterpart Rayon~\cite{rayon}

We end this section with an example of instruction reordering. Consider
Listing~\ref{lst:reordering}, and assume variables are initialized to 0.  When
ran sequentially (first B and then A), we would observe no difference if the
assignments of thread B were reordered; the end result would be that we print
\code{1}. However, when run concurrently thread B might decide to assign \code{f
= 1} first, and thread A will risk printing \code{0} instead of \code{1}. This
outcome was impossible in the sequential world.
\begin{figure}[ht]
\begin{lstlisting}[caption=Instruction reordering,label=lst:reordering]
// Thread A                      // Thread B
while f == 0 { }                 x = 1
print(x)                         f = 1
\end{lstlisting}
\end{figure}

A solution to this problem is to use a memory fence, in order to explicitly
disallow any reordering past it, as in Listing~\ref{lst:mem-fence}.
\begin{figure}[ht]
\begin{lstlisting}[caption=Memory fence for synchronization,label=lst:mem-fence]
// Thread A                      // Thread B
while f == 0 { }                 x = 1
fence()                          fence()
print(x)                         f = 1
\end{lstlisting}
\end{figure}

LLVM defines a memory model which is inspired by the C++11 memory
model~\cite{llvmmm}. The model includes six \emph{memory ordering constraints},
which are used in atomic operations. The orderings dictates how the compiler and
the CPU is allowed to reorder the instructions around operations the orderings
are used on.  We look at five of them here: \code{monotonic}, \code{acquire},
\code{release}, \code{acq\_rel}, and \code{seq\_cst}.  \code{monotonic} is the
weakest and offers no ordering constraints. Atomic operations with this ordering
differs only from regular operations in that the operation cannot be observed to
happen partially.  \code{acquire} and \code{release} are intended to work in
pairs, by loading with \code{acquire} and storing with \code{release} in the
same memory location.  An \code{acquire} load of X ensures that subsequent loads
will see all stores that happened before a \code{release} store to X.  This can
be used to implement a lock, where we \code{acquire} the lock before the
critical section, and \code{release} the lock after it.  Now we get the
guarantee that two critical sections cannot overlap, since the stores before the
\code{release} in the first section must be visible upon the \code{acquire}
load of the second. Note that operations \emph{are} allowed to be moved from the
outside to the inside of the critical section. \code{acq\_rel} is the
\code{release} ordering when used with a store, and \code{acquire} when used
with a load. \code{seq\_cst} is similar to \code{acq\_rel}, but it also
guarantees that all threads see all sequentially consistent operations is the
same order.

We can use these ordering constraints to improve our example. Fences are
expensive, so we would rather want to just make sure that there is an ordering
relationship between \code{f} and \code{x}. We can obtain this by using
\code{Acquire} and \code{Release} semantics on \code{f}, like in
Listing~\ref{lst:acqrel}. The \code{Release} in thread B ensures that the
assignment \code{x = 1} is not moved after the store to \code{f}, and the
\code{Acquire} load in thread A ensures that the access to \code{x} is not moved
above the load of \code{f}. This gives us the desired semantics.

\begin{figure}[ht]
\begin{lstlisting}[caption=Synchronization using orderings,label=lst:acqrel]
// Thread A                      // Thread B
while f.load(Acquire) == 0 { }   x = 1
print(x)                         f.store(1, Release)
\end{lstlisting}
\end{figure}

\subsection{Concurrency in Rust}

The Rust standard library contains multiple primitives for concurrency.
\code{std::sync} contains types such as \code{Arc} (atomic reference counted
smart pointer), \code{Condvar} (condition variable), \code{Mutex}, and
\code{RwLock} (Read-Write lock). The \code{std::sync::atomic} module contains
atomic primitives (but only for \code{bool}, \code{isize}, \code{usize}, and
\code{ptr}), and the \code{fence} function, which is a memory fence. All atomic
operations including the memory fence take an \code{Ordering} as the last
argument, which is the same memory orderings as LLVM uses (\code{monotonic} is
renamed to \code{Ordering::Relaxed}).

As seen in Section~\ref{sec:rust} Rust forbids on shared mutability, but atomic
are made for shared mutability. For this reason, the functions on the atomic
types does not take \code{\&mut self}, but \code{\&self}, even though the
operation do mutate the value. This is an example of a case where we use
\code{\&mut} and \code{\&} not to signal mutability, but to signal whether it
is safe to perform the operation concurrently on multiple threads.

The \code{Mutex} is yet another example of a type that uses the RAII pattern.
\code{Mutex::new} takes the value that the mutex is protecting, and
\code{Mutex::lock} returns a new type, \code{MutexGuard}, which implements
\code{Deref<T>} and which unlocks the mutex in \code{Drop}. Wrapping the data
the mutex protects in the Mutex type makes it impossible to use the data without
acquiring the lock.


\section{The ABA Problem\label{sec:aba}}

The ABA problem is one of the most known problems in concurrent programming,
especially within the topic of memory reclamation. The essence of the problem is
that there are logical changes to a structure that we cannot observe due to
hardware limitations. For instance we might read a memory location twice,
observing no change in the memory read. We might then conclude that no change
has taken place. This conclusion however, might not be valid.

Consider the following real world analogy: assume you have a opaque bottle that
is filled with water. If you leave the bottle somewhere and return to it after
some time there is no way to see whether anyone has been drinking your water, by
simply inspecting the bottle from the outside. Someone might have taken the
bottle, drunk the water, and put the bottle back as it were. Even worse, someone
might have replaced your bottle with an identical bottle filled with bees.
Another example comes from in lock-free lists. Consider a linked list, where
nodes have a \code{next} pointer to the next element in the list.  By reading
the \code{next} field on a node twice at times $t\sb{0}$ and $t\sb{1}$, and
observing no change, we might conclude that the next node is the same. However,
the node at $t\sb{0}$ may have been removed from the list, its memory reclaimed
for a new node, which is then inserted at the same place before $t\sb{1}$. This
can be disastrous and cause problems such as double deletes.

Instances of the ABA problem is easily prevented if we have atomic operations
that can check for change in the memory, even if it is changed back to the
initial state. An example of such operations are the \code{load-link} and
\code{store-conditional} instructions (\code{LL/SC}). The \code{SC} instruction
will only store a new value if the memory has not been written to after read by
\code{LL}. Current implementations on \code{LL/SC} are \emph{weak}, in the sense
that the store may fail even though it is never touched, but when another value
on the same cache line is written to. Architectures supporting \code{LL/SC}
include ARM, PowerPC, and RISC-V.

Another instruction that helps with the ABA problem is double compare-and-swap,
or \code{DCAS}, which is a two \code{CAS} operations that both must succeed for
any value to be updated. \code{DCAS} is not supported by any architecture. We
note that this is a different operation than double \emph{word} compare-and-swap
(\code{DWCAS}), which is a regular \code{CAS} that writes double word values.

Without proper hardware support there is no one good solution to the ABA
problem, but most instances of the problem are manageable. Solution ideas include
\emph{tagging}, in which we tag the value swapped such that the second reading
observes a change in tag; \emph{indirection} in which we use an intermediate
node, such that the intermediate node changes, while the data node is still not
observed to have changed; and \emph{deferred reclamation} where we wait for
``safe periods'' in which we know that ABA problems cannot occur. In the
following chapter we look at reclamation schemes which utilizes all three of
these solution ideas.


\section{Related Work\label{sec:related-work}}

While Rust does not have native support for concurrent memory reclamation in its
standard library, third party crates for this does exist. The most notable begin
Crossbeam~\cite{crossbeam}, an open source umbrella project for concurrency in
Rust. Crossbeam includes an implementation of EBR (Section~\ref{sec:ebr}), which
is included in the experimental results of this report. Work on HP in the Rust
community has also been done~\cite{ticky:hp}, and the Crossbeam project is as of
13th of December 2017 looking into possibilities of an
implementation~\cite{crossbeam-hp}.

Other languages are also turning their attention to memory reclamation. A
proposal for including hazard pointers and RCU (read-copy-update, not covered in
this report) into the C++ standard library was released in November 2017
\cite{cpp:mr}. There is also ongoing work in managed languages, despite the
presence of a garbage collector. An example is Project
Snowflake~\cite{project-snowflake-non-blocking-safe-manual-memory-management-net}
which combines ideas from both EBR and HP on the .NET platform.



\chapter{Memory Reclamation\label{ch:memory-reclamation}}

Dynamic memory allocation is a feature supported by the operating system. The
allocated memory must, sooner or later, be returned, or else there will
eventually be no more memory to allocate. The problem of concurrent memory
reclamation is all about finding out when it is safe to return memory to the
OS\@. The main source of problems in this field is that the threads may read
data, and then be preempted for an arbitrary long time; when it gets execution
time again, we must somehow make sure that the memory addresses it obtained are
either still valid, or that it can somehow validate them. Since threads can be
preempted at any point in execution, validation is a hard problem.

Most of the mainstream programming languages today has a large runtime which
employs a \emph{garbage collector}. The job of the garbage collector (GC) is to
find all memory that is no longer in use, and free it. A type of tracing garbage
collector is the \emph{mark-and-sweep GC}, which marks all pointers reachable
from some ground set of the program, sweeps the memory and frees the memory that
was not previously marked. This process involves pausing the program execution,
and is also dependent on the runtime thread.  One of the strengths of concurrent
systems is fail tolerance: thread may fail spuriously, but the system as a whole
makes progress regardless. It is clear that the runtime is a single point of
failure for systems employing one. It is therefore interesting to look at memory
management schemes that does not use a runtime with a garbage collector.

The field of concurrent memory reclamation is an active one, and a lot of
different schemes have emerged the recent years. We will look at \emph{Reference
Counting} (RC), \emph{Epoch Based Reclamation} (EBR), \emph{Hazard Pointers}
(HP), and try to give a highlight recent developments in the field.

\section{Reference Counting}
Reference counting (RC) is a natural solution for memory reclamation. It was
introduced in 1960 by G. E.  Collins\cite{collins1960method}, where it was used
for collecting nodes of a linked list.  The idea is that we count the number of
references to data, so that we can tell if we are holding the only reference to
some data. When we no longer need this reference, we know it is safe to reclaim
the memory the reference points to, since no other reference to that memory
exists. The primary downsides of RC is that it is rather expensive, and that a
na\"\i{}ve implementation does not reclaim cycles. Today reference counting is
still used, although it is unusual to have it be the primary mechanism for
memory management, due to its performance overhead.

Atoimc reference counting (ARC) is RC using atomic variables, and is a natural
extension of RC\@. However, the na\"\i{}ve implementation is not correct:
consider two threads operating on some \code{Rc<T>}. When thread $A$ want to
create a new reference to the data, it increments the count in the \code{RC}
object. Upon destruction, the count is decremented and the data is freed if the
count is 0. However, it is possible that thread $B$ has a reference to the RC
object and that it got preempted right before incrementing the count. Then the
whole object gets freed by thread $A$, since the count is 0, and when thread $B$
gets execution time again, it has a pointer to freed memory which it indents to
read.

A way to mitigate this problem is by indirection: we can use intermediate
\code{Rc} nodes which are the counter and a pointer to the actual data. The
intermediate nodes are never free'd, and by \code{CAS}ing the count to a
sentinel value upon destruction of the data, thread $B$ can detect that it is
about to read free'd memory and abort its operation. By allocation the \code{Rc}
objects with a memory arena and freeing them in bulk, this might be acceptable
for certain problems, as the data itself is not leaked, but only the \code{Rc}
nodes, which may be comparably small.

Despite the problem of atomic reference counting, there are still use cases for
it. A thread $A$ may create an \code{Arc} object, and make a copy of its
reference to it, incrementing the count, and only \emph{then} pass it to another
thread. This avoids the problem in the previous paragraphs, since the only
threads that need to increment the reference count is already holding onto
another reference, thus making it impossible that the count reaches zero before
we get to increment it. When all threads have dropped their reference, the
count will drop to zero, and the \code{Arc} will be free'd, not risking that any
other thread is just about to increment its count.
\clearpage
\section{Epoch Based Reclamation\label{sec:ebr}}

Epoch Based Reclamation (EBR) was introduced by Fraser
in~\cite{fraser2004practical}. It is a reclamation scheme based on the
observation that most programs have no references to internal data structure
memory in between of operations on the structure. The time interval in between
operations on the data structure are therefore safe-points (also called grace
periods) for memory reclamation to occur, since we do not risk invalidating any
data that other threads are using in this period. EBR uses the concept of an
\emph{epoch}, a global timestamp which we use to find out when it is safe to
reclaim retired memory. The epoch is a global counter. In addition we have a
global list with one entry for each running thread, which the threads use for
broadcasting their state, which includes the last epoch they read as well as
whether they are currently performing an operation. We call a thread performing
an operation \emph{pinned}, and the action of marking and unmarking
\emph{pinning} and \emph{unpinning} the thread.

When starting an operation a thread reads the global epoch, stores it in its
entry, and pins the thread. Upon retiring memory the thread marks the memory
with the global epoch and puts it in a \emph{limbo list}. Every once in a while,
the threads try to increment the epoch, which succeeds if all current pinned
threads have seen the current epoch. Note that we only have to look through the
thread entries once: if another thread is pinned while we are searching, it will
read the current epoch, and cause no problems for us. This requirement for
epoch incrementation means that all threads that have references to memory we
might want to free is either in the current or the previous epoch. Lastly, after
incrementing an epoch to $e$ we know that garbage that was added in epoch $e-2$
is safe to be freed.

Note that it is important that the thread inserting into the limbo list uses the
global epoch, and not the epoch it read when it was pinned. If we use the
previously read epoch, we may run into the following scenario:
\begin{enumerate}
  \item $A$ pins the thread at $e=5$, and wants to remove $O$ from the data structure.
  \item $B$ increments the epoch to $e=6$, and obtains a reference to $O$.
  \item $A$ unlinks $O$ from the data structure, and adds it to the limbo list,
    with $e=5$. $A$ unpinns, and increments the epoch to $e=7$.
  \item It is now safe, by our rules, to free $O$, although $B$ is still holding
    a reference to it.
\end{enumerate}
By reading the global epoch before pushing to the list we avoid this problem,
since $O$ is unlinked from the data structure before reading the epoch. This
makes it impossible for $B$ to have incremented the epoch, and \emph{then} get a
reference to $O$, without $A$ reading the incremented epoch.

EBR is very popular, due to its extremely low overhead.  However, there are
still a few challenges with EBR\@. A problem is that we are not allowed to keep
references to data across operations, since the thread must be pinned while we
are using the references. A natural way to mitigate this constraint is to leave
the thread pinned. However, this will stop the advancement of the global epoch,
and thus effectively halting the memory reclamation. An immediate consequence of
this is that EBR is not lock-free, which is not acceptable for all use cases.

\section{Hazard Pointers\label{sec:hp}}

Hazard pointers were introduced by Michael in~\cite{michael2004hazard}.  The
paper formalizes hazardous pointers, and includes a proof of correctness. We
will settle for a informal view of them. It is based on the observation that in
most operations on data structure we only need a small constant number of
references to memory that is shared between running threads. The technique
exploits this by allowing each thread to register the pointers, called
\emph{hazard pointers}, the thread wants to use, but which it cannot be sure are
valid. We call potentially invalid pointers \emph{hazardous}. The number of
pointers we need varies with the algorithm performed, but a typical value is one
or two.

After reading a hazardous pointer the thread registers it as one of its hazard
pointers. It then have to \emph{validate} that the pointer is still in the data
structure, as it might have been removed in between the initial read and the
hazard registration. When we want to free memory we look through the hazard
pointers of all running threads. We note that again, as with EBR, a single pass
through this list is sufficient: the object is unlinked before searching, so if
a thread has a reference to it, but is yet to register it as one of its hazard
pointers, then it will fail validation.

If the memory is registered in a thread, we cannot immediately free the memory.
We now have two options: wait for the thread to finish, or defer the
deallocation.  By waiting on the thread, we are relying on that the other thread
will ever deregister the pointer. Hence, we give up lock-freedom, as this is
prone to deadlocking. It has, however, very low overhead, and will be very fast
assuming all threads are fairly scheduled and have similar work load.  Deferring
the deallocation is a safer option, although it have a higher overhead, eg.\ of
pushing the pointer onto a queue. We would then occasionally visit the queue and
see if any of the pointers in it have been deregistered by all threads.

A challenge in usage of HP is that we need to identify which pointers in our
algorithms are hazardous. In comparison, we have no such concerns in EBR, in
which we only need to register memory as garbage when we remove it from the data
structure (we do need to make sure that this memory is only registered by a
single thread). Another challenge is that of validation, as there is no generic
way to do this. For most structures there is an obvious way of doing this. For
instance, in a queue we can validate the front element by reading the
\code{head} pointer again, observing that it has not changed. However, it still
requires local knowledge of the data structure in question.

\section{Other Schemes\label{sec:other-schemes}}

Due to time constraints we have chosen to focus on the two most known
reclamation schemes, EBR and HP\@. However, there exists a wide range of other
schemes, many of which have emerged only in the last recent years. We give a
very quick overview of selected schemes in the field. We encourage readers to
read the respective papers.

Brown introduces in~\cite{brown2015reclaiming} the \emph{DEBRA} and \emph{DEBRA+} systems,
which are based on ideas similar to that of EBR\@, but they also employ hazard
pointers. DEBRA+ uses additional OS specific features such as signals and
\code{longjmp} offered by POSIX compliant operating systems.

\emph{Drop the Anchor} is presented in~\cite{Braginsky:2013:DAL:2486159.2486184} by
Braginsky, Kogan, and Petrank, in which threads suspected to be stuck are marked
as such. This way we do not have to consider threads that may sleep for a long
time when reclaiming, which saves a lot of trouble. It employs ideas from HP,
like threads having a local buffer of deferred freed memory, and EBR by using
timestamps.

Cohen and Petrank proposed in~\cite{cohen2015efficient} a scheme in which the
memory manager allows threads to infrequently access reclaimed memory.

\emph{QSense} was introduced in~\cite{balmau2016fast}. It uses the
fast-path/slow-path pattern, in which the scheme in the fast path is EBR
inspired, and the slow-path is HP inspired. The system handles automatically the
switch between the two states.

Alistarh et.\ al presented \emph{StackTrack} in~\cite{alistarh2014stacktrack}
which is based around hardware transactional memory (HTM; not covered in this
report). They note the system can also be implemented using software
transactional memory, although HTM is essential for performance. In 2017 Alistar
et.\ al introduced \emph{ForkScan}~\cite{alistarh2017forkscan}, which employs signaling
and copy-on-write semantics offered by high performance implementations by
modern operating systems to process memory snapshots.


\chapter{Methodology\label{ch:methodology}}

In this chapter we will look at all the practicalities of implementing systems
for concurrent memory reclamation. We start by look at implementation of both the
data structures we need and the reclamation schemes; then we discuss testing of
concurrent system; finally we look at benchmarking the system.

\section{Implementation}

Now that we have an idea of \emph{what} we want to implement we must change our
mindset into \emph{how} we want to implement it. We make use of Rust's features
in order to make the system easy to use and hard to misuse.  As there is often a
disconnect between the description of a system and its implementation we risk
repeating parts of Chapter~\ref{ch:memory-reclamation} in this chapter. We note
that due to time limitations of this project the implementation is yet to be
properly optimized. The data structures were first implemented without any
regard for memory reclamation.  Allocated memory were simply leaked when no
longer needed.


\subsection{Atomics\label{sec:atomics}}

As mentioned in Section~\ref{sec:rust}, it is idiomatic Rust to unitize the type
system in order to help ourselves. We have made new atomic pointer types in addition
to those in the standard library in order to capture wanted semantics. Initially
these were copied from the Crossbeam project\cite{crossbeam-msqueue}, but they have been adjusted as
needed. Most notably a \code{HazardPtr} type was created, which handles
registration and deregistration of the hazard pointer automatically.


\subsection{Data Structures\label{sec:data-structures}}

For comparing concurrent memory reclamation in a meaningful way we need shared
memory, organized in some data structure. We have chosen to implement two of the
most commonly seen concurrent data structures: the Queue and the List. Both
implementation are well known in the field of concurrent data structures. We
chose to implement these relative simple data structures due to time constraints
of the project.


\subsubsection{Queue}

The queue implemented is a Michael-Scott Queue, as described
in~\cite{michael1996simple}. The \code{Queue} and \code{Node} structs, as well
as the signatures for the public functions are listed in
Listing~\ref{lst:msqueue}. The implementation of \code{push} and \code{pop} is
heavily inspired of the implementation from Crossbeam.  We support
\code{pop\_if}, since we use this in the implementation of EBR.  \code{push}
allocates the memory needed for the node, which in micro benchmarks have been
shown to take the majority of the time: allocation averaged at 54ns, while the
rest of the procedure averaged at 30ns.  A natural optimization of this problem
is to allocate nodes from a memory arena or similar, such that the allocation
overhead is amortized over multiple calls to \code{push}. However, this
seriously increases the complexity of the memory management schemes.

\begin{figure}[ht]
\begin{lstlisting}[caption=Structs for the Michael-Scott Queue,label=lst:msqueue]
pub struct Node<T> {
    data: ManuallyDrop<T>,
    next: Atomic<Node<T>>,
}
pub struct Queue<T> {
    head: Atomic<Node<T>>,
    tail: Atomic<Node<T>>,
}
impl<T> Queue<T> {
    pub fn new() -> Self;
    pub fn push(&self, T);
    pub fn pop(&self) -> Option<T>;
    pub fn pop_if(&self, Fn(&T) -> bool) -> Option<T>;
    pub fn is_empty(&self) -> bool;
}
\end{lstlisting}
\end{figure}

The \code{ManuallyDrop} type makes sure the data it wraps, in this case
\code{T}, is not automatically dropped when the node is dropped. We would rather
have the receiver of the data drop the values in the queue. \code{ManuallyDrop}
also makes it possible to \emph{do} drop the data. This is used in the
\code{Drop} implementation of the \code{Queue} itself, since we would like the
values in the queue to be cleaned up correctly when the queue is destroyed.

\subsubsection{List\label{sec:impl-list}}

The list we have implemented is based on the list presented by Michael
in~\cite{michael2002high}. The implementation is similar to the Michael-Scott
Queue in multiple ways. For instance, we support a \code{remove\_front}
operation, which is almost identical to \code{Queue::pop}. However, the default
insertion operation of the List pushes the inserted element to the beginning of
the list, such that \code{List::insert} and \code{List::remove\_front} has LIFO
semantics. We also support removals for values which are comparable (implements
\code{PartialEq}). This procedure complicates the implementation: consider the
list in Figure~\ref{fig:list-remove}. We want to remove node B, so we swing
\code{A.next} from B to C. However, at the same time, a thread might insert a
new node X in between B and C. When we now change \code{A.next} to \code{C}, we
have removed two items: B and X. This is an example of the ABA problem, which we
saw in Section~\ref{sec:aba}. Similar problems may arise even without the
possibility of middle-of-the-list insertions.

\begin{figure}[ht]
  \begin{subfigure}[b]{\textwidth}
      \centering
      \begin{tikzpicture}
        \node [lnode,label={A}] (A)              {\code{data} \nodepart{second} \code{}};
        \node [lnode,label={B}] (B) [right of=A] {\code{data} \nodepart{second} \code{}};
        \node [lnode,label={C}] (C) [right of=B] {\code{data} \nodepart{second} \code{}};
        \draw[-latex] ($ (A.west) - (0.5,0) $) -- (A.west);
        \draw[ptr] ($ (A.east) - (0.25,0) $) -- (B.west);
        \draw[ptr] ($ (B.east) - (0.25,0) $) -- (C.west);
        \draw[ptr] ($ (C.east) - (0.25,0) $) -- ($ (C.east) + (0.5, 0) $);
        \draw[-latex,color=lightgray] ($ (A.north) + (0.5,0) $) to[out=45,in=135] ($ (C.north) -
          (0.5, 0) $);
      \end{tikzpicture}
      \caption{The initial list. When removing B we swing the \code{}
      pointer over to C.\label{fig:list-remove-a}}
  \end{subfigure}

  \begin{subfigure}[b]{\textwidth}
      \centering
      \begin{tikzpicture}
        \node [lnode,label={A}] (A)              {\code{data} \nodepart{second} \code{}};
        \node [lnode,label={B}] (B) [right of=A] {\code{data} \nodepart{second} \code{}};
        \node [lnode,label={X}] (X) [right of=B] {\code{data} \nodepart{second} \code{}};
        \node [lnode,label={C}] (C) [right of=X] {\code{data} \nodepart{second} \code{}};
        \draw[-latex] ($ (A.west) - (0.5,0) $) -- (A.west);
        \draw[ptr] ($ (A.east) - (0.25,0) $) -- (B.west);
        \draw[ptr] ($ (B.east) - (0.25,0) $) -- (X.west);
        \draw[ptr] ($ (X.east) - (0.25,0) $) -- (C.west);
        \draw[ptr] ($ (C.east) - (0.25,0) $) -- ($ (C.east) + (0.5, 0) $);
        \draw[-latex,color=lightgray] ($ (A.north) + (0.5,0) $) to[out=45,in=135] ($ (C.north) -
          (0.5, 0) $);
      \end{tikzpicture}
      \caption{Another thread inserts a new node X between B and C.\label{fig:list-remove-b}}
  \end{subfigure}

  \begin{subfigure}[b]{\textwidth}
      \centering
      \begin{tikzpicture}
        \node [lnode,label={A}] (A)              {\code{data} \nodepart{second} \code{}};
        \node [lnode,label={\color{lightgray}B},color=lightgray] (B) [right of=A]
          {\code{data} \nodepart{second} \code{}};
        \node [lnode,label={\color{lightgray}X},color=lightgray] (X) [right of=B]
          {\code{data} \nodepart{second} \code{}};
        \node [lnode,label={C}] (C) [right of=X] {\code{data} \nodepart{second} \code{}};
        \draw[-latex] ($ (A.west) - (0.5,0) $) -- (A.west);
        \draw[ptr-g,color=lightgray] ($ (B.east) - (0.25,0) $) -- (X.west);
        \draw[ptr-g,color=lightgray] ($ (X.east) - (0.25,0) $) -- (C.west);
        \draw[ptr] ($ (C.east) - (0.25,0) $) -- ($ (C.east) + (0.5, 0) $);
        \draw[ptr] ($ (A.north) + (0.37,-0.26) $) to[out=45,in=135] ($ (C.north) -
          (0.5, 0) $);
      \end{tikzpicture}
      \caption{End result. B and X is no longer reachable from the head of the
        list.\label{fig:list-remove-c}}
  \end{subfigure}
  \caption{List removal which removes two elements when there is a concurrent
  insertion. This happens because the \code{compare-and-swap} operation
  performed on \code{A.next} is succeeds since the previous value is not
  changed.  However, the new value is logically changed, since C is no longer
  the \code{next} value of B, although there is no way for the \code{CAS} to
  detect that.\label{fig:list-remove}} \end{figure}

Our solution to this problem is simple and effective: we \emph{tag} the node we
want to remove, such that other threads do not add a new node after it. We
exploit the fact that memory addresses are aligned, meaning that the address of
an object in memory is divisible by some number. Objects are typically aligned
by their size, such that \code{u32}s are only on addresses where the four least
significant bits are zero. Bytes are ``aligned'' on 1 byte boundaries,
effectively meaning they are not aligned at all. Since we know that the pointers
we have points to larger objects, the LSB of the pointer itself is free. We use
this to tag the node. A consequence of tagging is that all operations must check
the \code{next} pointers for tags, and abort their operation if it contains one,
since the node they are currently on are being removed. This is especially a
problem when using hazard pointers, as the next node is not protected, and may
have been both removed from the list as well as deallocated.


\subsection{Epoch Based Reclamation}

Our implementation of EBR has both global state and thread local state. The data
in the global state contains three things: the epoch, which is an
\code{AtomicUsize}; a queue of retired garbage, which are tuples of pointers to
memory that we want to free and the epoch when the memory was retired; and a
list of thread markers, \code{List<ThreadMarker>}. This is all stored in the
\code{GlobalState} struct. We use the third-party crate \code{lazy\_static}
which contains a macro for lazy-initialized static variables. The thread local
data is stored in the \code{LocalState} struct, and is initialized using the
\code{thread\_local!} macro, which is in the Rust standard library. The thread
local data contains three things: a pointer to the threads list entry, a counter
of how many times the thread has been pinned, and buffered garbage (see below).
Figure~\ref{fig:ebr-impl} shows graphically the global and local data.

\begin{figure}[hb]
  \begin{tikzpicture}[ntitle/.style={node distance=1cm, font=\footnotesize}]
    \node (global-title) [anchor=west] {\normalsize\textbf{Global State}};

    \node [ntitle] (epoch-title) [below=of global-title.west, anchor=west] {Epoch:
      \code{AtomicUsize}};
    \node [draw=black!30, fit={(epoch-title)}]
      (epoch-box) {};

    \node [ntitle] (pins-title) [below=of epoch-title.west, anchor=west]
      {Thread Pin List: \code{List<ThreadPinMarker>}};
    {%
      \node [lnode,node distance=0.7cm]  (pins-f-a)  [below=of pins-title.west,
        anchor=west] {\code{marker} \nodepart{second} };
        \node [lnode,node distance=0.7cm]  (pins-f-b)  [right=of pins-f-a]
          {\code{marker} \nodepart{second} };
        \node [lnode,node distance=0.7cm]  (pins-f-c)  [right=of pins-f-b]
          {\code{marker} \nodepart{second} };
      \draw[ptr] ($ (pins-f-a.east) - (0.25,0) $) -- (pins-f-b.west);
      \draw[ptr] ($ (pins-f-b.east) - (0.25,0) $) -- (pins-f-c.west);
    }
    \node [draw=black!30, fit={(pins-title) (pins-f-c)}] (pins-box) {};

    \node [ntitle] (garbage-title) [below=of pins-f-a.west, anchor=west]
      {Garbage Queue: \code{Queue<(usize, Bag)>}};
    {%
      \node [lnode,node distance=.7cm]  (garbage-f-a)  [below=of garbage-title.west,
        anchor=west] {\code{(10, bag)} \nodepart{second} };
        \node [lnode,node distance=0.7cm]  (garbage-f-b)  [right=of garbage-f-a]
          {\code{(10, bag)} \nodepart{second} };
        \node [lnode,node distance=0.55cm]  (garbage-f-c)  [below=of garbage-f-a]
          {\code{(11, bag)} \nodepart{second} };
        \node [lnode,node distance=0.7cm]  (garbage-f-d)  [right=of garbage-f-c]
          {\code{(12, bag)} \nodepart{second} };
      \draw[ptr] ($ (garbage-f-a.east) - (0.25,0) $) -- (garbage-f-b.west);
      \draw[ptr] ($ (garbage-f-b.east) - (0.20, -0.05) $) --
        ($ (garbage-f-b.east) - (0.20,0.5) $) --
        ($ (garbage-f-c.north) - (0,-0.30) $) --
        (garbage-f-c.north);
      \draw[ptr] ($ (garbage-f-c.east) - (0.25,0) $) -- (garbage-f-d.west);
    }
    \node [draw=black!30, fit={(garbage-title) (garbage-f-d)}] (garbage-box) {};

    \node [draw=black, dotted, fit={(global-title) (pins-box) (garbage-box)}]
      (global-box) {};


    \node (local-title) [anchor=west, right=of global-title, shift={(3.5, 0)}]
      {\normalsize\textbf{Thread Local State}};

    \node [ntitle] (pinptr-title) [below=of local-title.west, anchor=west]
    {Thread Pin: \code{*const ThreadNodePtr}};
    \node [draw=black!30, fit={(pinptr-title) }]
      (pinptr-box) {};

    \node [ntitle] (pincount-title) [below=of pinptr-title.west, anchor=west]
    {Pin Count: \code{usize}};
    \node [draw=black!30, fit={(pincount-title) }]
      (pincount-box) {};

    \node [ntitle] (bag-title) [below=of pincount-title.west, anchor=west]
    {Garbage Bag: \code{Bag}};
    {%
      \node [lnode,node distance=.7cm,
      rectangle split parts=5
      ]  (bag-fig)  [below=of bag-title.west,
        anchor=west] {\nodepart{one}
          \code{garbage} \nodepart{two}
          \code{garbage} \nodepart{three}
          \code{garbage} \nodepart{four}
          \code{empty} \nodepart{five}
          \code{empty}
        };
    }
    \node [draw=black!30, fit={(bag-title) (bag-fig)}]
      (bag-box) {};

      \node [anchor=west, draw=black, dotted, fit={(local-title) (pinptr-box)
      (pincount-box) (bag-box)}]
      (local-box) {};
  \end{tikzpicture}
  \caption{The global and thread local data for the implemented EBR scheme. The
    global state contains the current Epoch, a list of thread pin markers, and a
    global garbage queue, in which garbage bags with an epoch is deferred for
    reclamation.  The thread pin in the thread local state points to that
    threads entry in the thread pin list. ``Pin Count'' is incremented each time
    the thread calls \code{pin}, and we use this number to choose when to
    reclaim memory from the global garbage queue. The thread local garbage bag
    is buffered garbage, so that we amortize the garbage queue overhead, since
    the nodes in \emph{that} list also have to be deferred for reclamation,
    using itself.\label{fig:ebr-impl}}
\end{figure}

Memory we want to free is hidden behind the \code{Garbage} struct, which
abstracts away the logic of calling the right destructor when we free the
memory. This is needed since \code{LocalState::add\_garbage} takes an
\code{Owned<T>}, an owned pointer to \emph{any} data type, as we want to be able
to use the same reclamation system for all types of memory. The type information
is then lost for the rest of the garbage pipeline. However, upon destruction we
need to know the type of the data we are dropping. The way we have solved this
is by moving the data to a closure, so that the closure keeps track of the type
of the data. The closure is never invoked, but upon dropping the closure all of
the values moved to it are dropped. A advantage of this solution is that it is
simple to implement. A disadvantage is that the closure needs to be heap
allocated, which increases the overhead of adding garbage. We note that
Crossbeam have mitigated the problem of heap allocation by stack allocating
closures that are ``small enough''.

We note that both the thread entry list and garbage queue themselves must be garbage
collected, and that we use EBR on them. This poses a problem: for each
garbage in the list, we need a node. However, that node itself will be garbage
when it is popped from the list, so we need to push it into the garbage list,
which makes a node, etc. Our solution to this is to chunk up garbage in chunks
of a constant size, using the \code{Bag} struct. Thus the garbage queue is a
\code{ebr::queue::Queue<(usize, Bag)>}. The chunking is done thread locally, which also
lowers synchronization overhead, since fewer elements are shared between threads.

When using the collections backed by EBR, we must first obtain a \code{Pin},
which is a proof that the current thread is pinned. All methods on any
collection requires a pin as the last argument, even though the methods may not
actually use the pin in the method body. There is no way to obtain the pin
directly: users \emph{must} call the \code{pin} function, and pass in a closure
which is then given the pin as an argument. Listing~\ref{lst:pin-ex} shows
example usage on a queue. This design decision is made in order to discourage
users to grab a pin and keep it for a long amount of time, as this will
effectively stop the memory reclamation. Having a closure also makes it very
clear when the thread stops being pinned. Internally in the \code{ebr} module,
there is a method called \code{Pin::fake}, which makes a pin without actually
pinning the thread. This is an optimization used when we know that we have
exclusive access on certain memory.

\begin{figure}[ht!]
\begin{lstlisting}[caption=Example usage of the \code{pin} fucntion,
label=lst:pin-ex,numbers=none]
let queue = Queue::new();
pin(|pin| {
  queue.push(42, pin);
});
\end{lstlisting}
\end{figure}

The \code{pin} function is also where we reclaim the memory. Before calling the
closure passed in, we read the global epoch, and increment our local pin
counter. Every $n$th call to \code{pin}, we try to increment the epoch. For
incrementation to occur, all the pinned threads must have read the current
epoch.  If we succeed at incrementing the epoch we also free as much garbage as
allowed from the global garbage list. This scheme is in a sense fair, in that
threads that pin a lot supposedly create a lot of garbage, and these threads
will also have to clean up once in a while. It does however suffer from the fact
that a single thread is set to clean up all the garbage from potential all
other threads.  This increases the variance of the overhead of EBR, which is
confirmed experimentally in Chapter~\ref{ch:results}. An idea for an
alternative implementation could register globally that all threads are to
collect a smaller amount of garbage if they enter the \code{pin}
function\footnote{Norwegians will recognize this concept as \emph{dugnad}}.

The \code{Pin} struct contains a method for retiring garbage, aptly called
\code{add\_garbage}. This method handles the thread local caching of garbage
objects into a \code{Bag}, and pushes the bag with the current epoch into the
global garbage queue when the bag is full. It does not support flushing the
current \code{Bag} (that is, adding it to the global queue even when it is not
full), although this is trivial to implement. Typical usage of
\code{add\_garbage} is to make a node unreachable by its data structure, read
the data needed from it, and then call \code{add\_garbage} with an
\code{Owned<Node<T>>}.  \code{pin} and \code{Pin::add\_garbage} are the only two
methods a user of EBR needs to use.

The thread entry list is a potential memory problem: each thread makes an entry
the first time \code{pin} is called, and this entry is only removed if the
thread shuts down gracefully. Thus, if the system keeps creating threads which
crashes right after calling \code{pin}, the list will quickly grow, and its
elements will never be removed, which will leak memory and severely increase the
overhead of EBR. This problem can be mitigated by removing the entry from the
list if the thread shuts down gracefully; however this is still a problem if
threads crashes. One solution could be to mark the entries with a timestamp, and
remove entries that have been inactive for too long, as well as to set the
threads thread local pointer to \code{null}. This creates further complications,
since thread may have simply been preempted for a long time, such that other
threads think it has crashed.  This problem has not been attempted solved as it
is mainly theoretical.\label{sec:thread-cleanup}

\subsection{Hazard Pointers}

The scheme for hazard pointers is simpler than that of EBR.\@ Each thread have a
\code{ThreadEntry} which contains a fixed size array of hazard pointers. A
hazard pointer is stored as a \code{AtomicUsize}.  The number of hazard pointers
is stored in a constant. The entries are stored in a global list, and each
thread has a local pointer to its entry in the list. The \code{Ptr} struct is
extended with a \code{hazard} method, which makes a \code{HazardPtr} struct.
\code{HazardPtr} utilizes the RAII paradigm by registering itself in its
constructor and deregistering itself in its \emph{drop} implementation. This way
users know that a pointer is protected if they have it as a \code{HazardPtr}. It
also contains methods for checking if its unique (that is, no other thread has
marked the same address as a hazard pointer).  Logically, a \code{HazardPtr} is
a proof that the pointer is safe to use by the thread. There is no abstraction
around the validation of the hazard pointers; this has to be done manually, and
was inlined in all places necessary.

As we saw in Section~\ref{sec:hp} we have primarily two ways
of handling freeing of data protected by hazard pointers: waiting and deferring.
We have implemented both, using Rusts \emph{conditional compilation}, in which
either variant is enabled through a flag passed to the compiler.

\subsubsection{Deferred Freeing} In deferred freeing, the implementation of HP
becomes more similar to that of EBR\@. We used the same \code{Garbage} struct as
in EBR to handle garbage types. We did not need to port over \code{Bag}, but
rather spun on contention of the hazard pointers for the nodes, as this queue
would have less contention.  The \code{hazard()} method on the atomic pointer
types from Section~\ref{sec:atomics} hijacks the program execution for freeing,
similar to the \code{pin} function in EBR\@. The hazard pointer is buffered
locally and then pushed into a global list. Occasionally we pop the hazard
pointer element of the queue if the pointer is unique. If the pointer is not
unique, we give up.

\subsubsection{Wait and Free}
The waiting scheme is simpler. When we want to free data protected by a hazard
pointer, we iterate over the global list of pointers and look for registered
pointers to the same address. If we find one, we spin until it is deregistered.
This gives up lock-freedom, but the overhead is minimal, assuming fairly
scheduled threads.


\section{Testing}
Testing is an important part of programming, even more so in concurrent
programming. Edsger Dijkstra famously said ``Testing shows the presence, not the
absence of bugs''\cite{buxton1970software}. In the concurrent world we should be
considered lucky if our tests even show the presence of the bugs they are
intended to catch.

Our approach to testing is simple. We write regular tests, and run them enough
times to make it improbable that any scheduling of the threads can cause
problems. While developing, we used unit tests. See Section~\ref{sec:rust-test}
for a primer on writing tests in Rust. Eventually, the benchmarks were ran as
tests, since the optimization done by the compiler revealed even more bugs.
Testing were primarily done on the development machines of the author, which are
all \code{x86} machines. Since we decided early on to only use the \code{SeqCst}
ordering, which removes most differences between x86 and ARM, we experienced no
bugs on ARM that was not already discovered on the \code{x86} machines.

The majority of the so-far found bugs in the system are memory bugs: use after
free bugs, double free bugs, illegal memory accesses, and memory leaks. Double
frees and illegal memory accesses are usually hard faults, such that the
programs execution stops, and we are notified that eg. \code{0x8} is indeed not
a valid memory address to read from.

Use after free bugs and memory leaks are more difficult to find. Here we found
great use in Valgrind\cite{valgrind}, a instrumentation framework, which
contains the tool \code{memcheck}. \code{memcheck} intercepts all memory
operations, tracks allocations and frees, and is able to report most, if not
all, memory problems. One minor problem with using Valgrind with Rust is that
Valgrind has incomplete support of the default allocator in Rust,
\code{jemalloc}\cite{jemalloc}. However, changing the allocator in Rust to the system
allocator, which Valgrind supports fully, is possible, and takes only four lines
of code in the \code{lib.rs} file of the crate. It does, however, require us to
use the nightly build of the compiler, as the API is not considered stable.

A downside with Valgrind is the decrease in runtime it imposes. Many hours have
been used waiting on Valgrind to hopefully reveal the illegal memory access that
happened earlier. In addition, since Valgrind is effectively a virtual machine,
the scheduling is different from when ran by the OS\@. This often made bugs
disappear, and we were forced to run the tests, or benchmarks, tens of times
(each of which already taking an unpleasant amount of time) in order to get
stack traces of the crashing program. While the situation is unfortunate, I do
not believe that there currently is a good solution at verifying a concurrent
systems\footnote{perhaps with the exception of a formal proof followed by a
careful implementation which hopefully matches the proof. This however is often
intractable for larger systems}.


\section{Profiling\label{sec:profiling}}

In order to better understand the trade-offs between EBR and HP we have written
benchmarks. We compare the performance overhead of HP and EBR by measuring the
total time it takes for some large operation to complete, using different number
of threads. The operations are \code{N} operations on the queue and list, which
were described in Section~\ref{sec:data-structures}. While the reclamation
system is designed for multi-threaded applications, it is also interesting to
look at the overhead when using only one thread since the overhead may be very
similar, which would be an indication that the scheme scales well.

While Rust do support testing, it does not support benchmarking on the stable
release, although there is a system similar to that of testing available on the
nightly release. There are also multiple third party benchmark suites, most of
which are based on the nightly benchmark system. As the implemented memory
reclamation schemes uses global state and thread local data, we must be careful
in how the benchmark suite operates, due to the problems noted in
Section~\ref{sec:thread-cleanup}. In addition, problems arise when benchmarking
the data structures without any memory reclamation, as the memory is quickly
exhausted, which either caused the process to crash or the memory to be swapped
to disk, destroying performance and invalidating the benchmark. Initially a
workaround was proposed, where the data structure operations saved a pointer to
the node they were leaking in a passed in memory location, but this was
difficult to generalize when using multiple threads. Note that this is
preferable to the caller allocating the node, since that would hide the
allocation cost in the benchmark (which is rather high). For these reasons, a
separate benchmark crate was developed.

The benchmark system is made for multi-threaded applications, and handles thread
spawning and joining internally. Upon setup the user defines the number of
threads that should be ran, as well as a state struct, which is given to all
closures. After setup the user passes a \emph{function pointer} to the system
(note, not a closure). The reason for this design decision is primarily
ergonomic, as we would like to avoid lifetime issues, but is is then also clear
to the user that the function passed may not mutate \emph{any} state. The
function must take as its only parameter a reference to the state struct, as
defined in the setup. The system runs the function repeatedly. We ran it 200
times on all platforms. These measurements were used to generate the graphs in
Chapter~\ref{ch:results} and Appendix~\ref{ch:benchmarks}. See
Listing~\ref{lst:benchmark} for a code sample of one of the benchmarks.

\begin{figure}[ht]
  \begin{lstlisting}[label=lst:benchmark,caption=Benchmarking \code{Queue::pop}
  using the self made benchmarking system with a variable number of threads.
  The closure passed to \code{.before} is executed before every call to the
  measured function and is used to initialize the state (in this case
  prepopulate the queue). The function passed to \code{.thread_bench}
  (\code{queue_pop}) is the measured function.]
pub fn queue_pop(num_threads: usize) -> bench::BenchStats {
    struct State {
        queue: Queue<u32>,
    }
    let state = State { queue: Queue::new() };
    let mut b = bench::ThreadBencher::<State, hp::JoinHandle<()>>::new(state,
                                                                       num_threads);
    b.before(|state| {
        while let Some(_) = state.queue.pop() {}
        for i in 0..NUM_ELEMENTS {
            state.queue.push(i as u32);
        }
    });
    fn queue_pop(state: &State) {
        while let Some(_) = state.queue.pop() {}
    }
    b.thread_bench(queue_pop);
    b.into_stats(format!("{}::queue::pop::{}", NAME, num_threads))
}
  \end{lstlisting}
\end{figure}

We set up multiple benchmarks, in order to try to find different scenarios in
which the schemes performs differently. We have five different benchmarks:

\begin{description}
  \item[\code{Queue::push}:] We measure the total time it takes to push \code{N}
    elements into a queue.
  \item[\code{Queue::pop}:] We measure the total time it takes to pop \code{N}
    elements out from a queue, which contains \code{N} elements. The threads
    repeatedly calls \code{pop} until \code{None} is returned.
  \item[\code{Queue::transport}:] We have two queues, one source and one sink.
    The source is pre-filled with \code{N} elements, and the sink is empty. The
    threads moves all elements from the source to the sink.
  \item[\code{List::remove}:] We pre-populate a list with \code{N} numbers in
    random order. Then each thread removes \code{N/t} unique elements, where
    \code{t} is the number of threads running. This ensures that all removals
    are succeeding and that the list is empty at the end\footnote{Note that if \code{t}
    does not divide \code{N} the queue is left with \code{N \% t} elements; this
    caused confusion when benchmarking systems where the number of threads is
    not a power of 2.}.
  \item[\code{List::real}:] We simulate a real world scenario, in which we
    perform different operations on the list. Which operation is performed is
    randomly selected and based on the following distribution: 40\%
    \code{insert}, 20\% \code{remove}, 20\% \code{remove\_front}, and 20\%
    \code{search}. The list is pre populated with \code{N} elements, in order for
    the removals not to fail often. The numbers which we insert, remove, or
    search for are also randomly generated from the range \code{[0, N)}.
\end{description}


\chapter{Experimental Results}\label{ch:results}

We have ran the benchmark suite on five different machines, in order to cover
CPU's from different vendors made for different use cases, hoping to uncover
differences in the reclamation scheme performance. As the CPU's have a very
different core and thread count, the number of threads in each benchmark is not
the same across CPU's, but are capped with the number of hardware threads
supported for the CPU\@. This means that the benchmarks for the lower end CPU's
only use up to 4 threads, while the ones for the highly parallel ARM CPU's use
up to 32. The CPU's we have tested on are shown in Table~\ref{tb:cpus}. All CPU's
are single socket.

\begin{figure}[b]
  \centering
  \caption{The CPU's on which we performed the benchmarks.\label{tb:cpus}}
  \begin{tabular}{c c c c c c}
    Vendor & CPU Name & Architecture & Clock Frequency & Cores & Hardware threads \\
    \hline
    Intel  & i7--7500U  & \code{x86}   & 2.70GHz & 2  &  4\\
    Intel  & i7--4770   & \code{x86}   & 3.40GHz & 4  &  8\\
    Intel  & E5--2620   & \code{x86}   & 2.00GHz & 6  & 12\\
    AMD    & Ryzen 1700 & \code{x86}   & 3.00GHz & 8  & 16\\
    Cavium & ThunderX   & \code{ARMv8} & 2.50GHz & 32 & 32
  \end{tabular}
\end{figure}

The results are grouped by CPU, each of which have their own section.  Within
each section each benchmark have a single bar graph, showing the ratio of the
time used to the time used without any memory reclamation on a logarithmic
scale. The x-axis shows the thread count. Note that the \code{queue::push}
benchmark is not present in this chapter. For the full data set, see Appendix\
\ref{ch:benchmarks}, which shows box plots for all performed benchmarks.

We note that one of the schemes we compare, Crossbeam, is a well developed open
source library, while the remaining, namely EBR, HP, and HP-Spin, are the
authors own implementation. It is therefore expected that Crossbeams offers
lower overhead of its operations. Crossbeam has no implemented concurrent list,
so it is only present in the queue benchmarks. In addition, the queue
implementations differ between Crossbeam and the author's, which causes
performance differences not strictly due to the presence of the memory
reclamation system.

The 7500U and the Ryzen are the authors personal machines. The 4770 was borrowed
from the department of computer science at NTNU\@. The 2620 was borrowed from
Webkom, the web-committee for Abakus, the student association for computer
science and communications technology at NTNU\@. The ThunderX was rented at
Scaleway\cite{scaleway}.

\newcommand{\benches}[2]{
  \begin{figure}[ht]
    \centering
    \begin{subfigure}{0.49\textwidth}
      \centering \footnotesize\code{Queue::Pop}
      \includegraphics[width=\textwidth]{plots/#1-b:queue_pop.pdf}
    \end{subfigure}
    \begin{subfigure}{0.49\textwidth}
      \centering \footnotesize\code{Queue::Transfer}
      \includegraphics[width=\textwidth]{plots/#1-b:queue_transfer.pdf}
    \end{subfigure}
    \\\vspace{0.25cm}
    \begin{subfigure}{0.49\textwidth}
      \centering \footnotesize\code{List::Remove}
      \includegraphics[width=\textwidth]{plots/#1-b:list_remove.pdf}
    \end{subfigure}
    \begin{subfigure}{0.49\textwidth}
      \centering \footnotesize\code{List::Real}
      \includegraphics[width=\textwidth]{plots/#1-b:list_real.pdf}
    \end{subfigure}
    \caption{Benchmarks relative to no memory reclamation for the #2}
  \end{figure}
}

\clearpage
\section{Intel\textregistered{} i7--7500U}

The i7--7500U is a high end CPU for the mobile segment. It has four hardware
threads and is hence the CPU with the fewest threads in the report. Our results
show that the implementation of EBR is less efficient than Crossbeams version.

In the queue benchmarks we see large differences between the schemes based on
EBR and the ones based on HP\@. The scheme with the most overhead is the
lock-free implementation of HP, which has twice the overhead compared to the
spinning version of HP\@. This could be because the operations performed are so
small that the large overhead of pushing the node to a queue effectively doubles
the work done.

In the list based benchmarks there is a large difference between EBR and HP\@.
This is expected, as HP has to register hazard pointers for each node when we
traverse the list, which is done in both \code{Remove} and \code{Real}. The
difference is somewhat less in \code{Real} using four cores, probably due to the
increased contention, but there is still a factor of 10 in difference. HP and
HP-Spin performs very similar here, as we see the opposite effect from in the
queue benchmarks, where the overhead of the queue push drowns in the slow
operations.

\benches{laptop}{Intel\textregistered{} i7--7500U}

\clearpage
\section{Intel\textregistered{} i7--4770}

The i7--4770 shows almost identical numbers as the 7500U. This is expected, as
both CPU's are x86 from Intel, and targets much of the same segment. As with the
7500U we see almost no difference in overhead as the thread count increases,
with the exception of \code{Queue::Pop} which do increase, and has the highest
overhead of that benchmark across all tested CPU's, especially using
Crossbeam.

\benches{gribb}{Intel\textregistered{} i7--4770}

\clearpage
\section{Intel\textregistered{} Xeon\textregistered{} E5--2620}

The E5-2620 is a CPU targeting the server segment. It is the oldest CPU in the
report, being released in Q1'2012. \code{Pop} shows us similar overhead as in
the previous Intel CPU's. \code{Transfer} however, shows an increased overhead
with the lower thread configurations, which reaches well past a factor of 10 with
one thread, but gradually decreases to 8, with 12 threads. That the overhead is
reduced with the increase in core count is also unique for the Xeon among the
Intel CPU's. It is surprising that only \code{transfer} and not \code{pop} have
this decrease, although it may be explained by the fact that \code{transfer}
allocates nodes, such that the ratio is smaller since the total time is larger.

Both \code{List} benchmarks also shows a slightly different performance profile,
where the overhead overall is lower than both the 7500U and the 4770.

\benches{server}{Intel\textregistered{} Xeon\textregistered{} E5--2620}

\clearpage
\section{Ryzen 7 1700}

The Ryzen is the tests only CPU from AMD, and has the second highest core count,
despite being a desktop CPU. One might expect AMD's focus on multi-core to
be manifested in the benchmark in the form of lower synchronization costs.
Despite this, the difference between EBR based schemes and HP based schemes for
\code{Pop}is the largest in the test. This is counter intuitive, as HP schemes
have a higher degree of synchronization. The results for the \code{List} tests
are similar to the 2620, with a little higher overhead with lower thread counts.

\benches{ryzen}{Ryzen 7 1700}

\clearpage
\section{Cavium ThunderX}

The Cavium ThunderX is the only ARM based CPU in the report. It is also the CPU
that produced the most surprising results. We see that certain benchmarks and
thread counts beat the baseline of no reclamation with all schemes except
\code{HP}. It is worth nothing that memory reclamation previously have been
shown to increase performance in the literature, such as\
\cite{brown2015reclaiming}. This may be due to the reduced memory footprint
which reduces the number of cache misses. We also see that the ThunderX scales
very on the \code{List} benchmarks, improving by a factor of 10 from 1 to 32
threads in \code{List::Remove}.

\benches{scaleway}{Cavium ThunderX}


\chapter{Conclusion\label{ch:conclusion}}

In this report we have looked at the problem of concurrent memory reclamation, a
field which importance increases as the core count of consumer hardware is
increasing. We have looked at two of the most popular schemes, Epoch Based
Reclamation (Section~\ref{sec:ebr}) and Hazard Pointers (Section~\ref{sec:hp}),
and implementations of them in the Rust programming language
(Chapter~\ref{ch:methodology}). We have also looked at the performance
implications of the two across multiple CPU's.  Neither implementation has
undergone careful optimizations, so the absolute performance numbers are not as
interesting as the general curves of the schemes, although the implementation of
EBR had similar overhead characteristics as Crossbeam in most benchmarks.

We used Rust as the implementation language. Rust has concurrency as one of its
main goals, and has shown to be a great language for constructing high level
abstractions as well as maintaining low level control. Rusts tooling has also
proved itself to be mature enough, even though we had to jump through some hoops
(like replacing the default allocator with the system allocator). I think Rust
shows great promise for systems programming, especially within the realm of
concurrency.


\section{Future Work}

This report is mainly a proof of research by the Author, and it thus does not
necessarily contain noteworthy novel parts. However, it does shed some light on
an increasingly popular and important field. I list topic I believe are natural
extensions of this project, or related projects.

This report only discusses implementation of two data structures, the list and
the queue. Other relevant structures to test against includes Skip-Lists,
Hash Maps, and Deques. It might also be interesting to look at the performance
implication of the reclamation schemes when using other primitives such as
channels.

Implementing other reclamation schemes in Rust could be interesting, as the
schemes can leverage the static checking that Rust does to a varying degree.  It
would be worth looking into implementations of any of the systems mentioned in
Section~\ref{sec:other-schemes} (or any of the many systems not mentioned), as
well as improvements of the implementations of EBR and HP, which already exists.
Performance, usability, and security are three important factors here.

As mentioned in Section~\ref{sec:related-work}, both Crossbeam and the C++
community is looking into generic and flexible implementations of hazard
pointers. This required careful engineering and planning, and could be a
valuable addition to both ecosystems.

With the emergence of support for hardware transactional memory it would be
especially interesting to have Rust implementations of reclamation schemes using
HTM\@. This would require either the Rust compiler to have support for HTM, or
third party crates to have inline assembly for the targets that support HTM\@.

Lastly, from an engineering standpoint, improved tooling for concurrent
programming would be much appreciated by the entire community. While Rusts
static checks and rules narrow the searches for bugs in the codebase down to
\code{unsafe} blocks, there still seems to be no good way of neither detecting
nor finding bugs that are due to concurrency.


\bibliographystyle{acm}
\addcontentsline{toc}{chapter}{Bibliography}
\bibliography{sources}

\begin{appendices} \chapter{Rust's Toolchain\label{ch:rust-toolchain}} The Rust
  toolchain consists of multiple tools. There is the compiler \rustc{}, the
  build tool and dependency manager \cargo{}, and the toolchain manager
  \rustup~\cite{rustup}. \rustup{} is the preferred way of installing Rust.
  \rustup{} also handles cross-compiling, i.e.\ compiling programs for a
  different architecture.

  A major part of Rust's ecosystem is \url{https://crates.io}, the package repository
  for Rust. This is where packages handled by \cargo{} is downloaded from, by default.

  \section{Hello World}
  We show how to install the Rust toolchain on a Unix based system.
  Installing \rustup{} is done by downloading and running an install script from
  \url{https://rustup.rs}:
  \begin{lstlisting}[language=Bash,numbers=none]
% curl https://sh.rustup.rs -sSf | sh
  \end{lstlisting}
  This installs \rustup{}, \cargo{}, and \rustc{}.
  Next we want to use \cargo{} to make a project. This is done by \code{cargo init}.
  \begin{lstlisting}[language=Bash,numbers=none]
% cargo init --bin <name-of-project>
  \end{lstlisting}
  This will create a directory of the name provided, containing two files:
  \code{Cargo.toml} and \code{src/main.rs}.
  The former is the configuration file of the project, which contains:
  metadata such as the project name, version, authors;
  build options such as optimization levels, debug levels, optional flags;
  and dependencies, with versioning and optional flags.
  The initial \code{Cargo.toml} may look like Listing~\ref{lst:cargo.toml}.
  The other file, \code{src/main.rs} contains the entry point of the program, as
  in Listing~\ref{lst:hello-world}.
  \clearpage{}

  \begin{figure}[ht]
  \begin{lstlisting}[language=,label=lst:cargo.toml,
  caption=A newly generated \code{Cargo.toml}]
[package]
name = "project-name"
version = "0.1.0"
authors = ["Martin Hafskjold Thoresen <martinhath@gmail.com>"]

[dependencies]
  \end{lstlisting}
\end{figure}

\begin{figure}[ht]
  \begin{lstlisting}[caption=Hello World in Rust,label=lst:hello-world]
fn main() {
    println!("Hello, world!");
}
  \end{lstlisting}
\end{figure}

  To run the program, we use \cargo{}:

  \begin{figure}[ht]
  \begin{lstlisting}[language=Bash,numbers=none]
% cargo run
   Compiling project-name v0.1.0 (file:///<path>/)
    Finished dev [unoptimized + debuginfo] target(s) in 0.68 secs
     Running `target/debug/project-name`
Hello, world!
%
  \end{lstlisting}
\end{figure}

  Cargo supports a project to build multiple executables, or no executables at all.
  For more information about cargo see~\url{http://doc.crates.io/index.html}.

  \section{\code{\#[test]} and \code{\#[bench]}}
  \label{sec:rust-test}
  The Rust toolchain supports both testing and benchmarking. To write tests we
  make an inline module named \code{test}, and conditionally compile it using
  \code{\#[cfg(test)]}. This makes the code to be ignored unless we are running
  the tests; this is done with \code{cargo test}. Listing~\ref{lst:cargo-test}
  shows an example test on a \code{Queue}. This is usually put in the same file
  at the \code{Queue} but this is only by convention, and not required.

  \begin{figure}[ht]
  \begin{lstlisting}[label=lst:cargo-test,caption=An example test in Rust]
#[cfg(test)]
mod test {
    use super::*;
    #[test]
    fn queue() {
        let mut queue = Queue::new();
        queue.push(1);
        queue.push(2);
        assert_eq!(queue.pop(), Some(1));
        assert_eq!(queue.pop(), Some(2));
    }
}
    \end{lstlisting}
  \end{figure}
  The output of \code{cargo test} might look like
  Listing~\ref{lst:cargo-test-output}.
  \begin{figure}[ht]
    \begin{lstlisting}[label=lst:cargo-test-output,caption=Sample output of
    \code{cargo test},language=,numbers=none]
running 47 tests
test ebr::atomic::tests::valid_tag_i8 ... ok
test ebr::atomic::tests::valid_tag_i64 ... ok
test ebr::bench::pin ... ok
test ebr::queue::bench::push ... ok
test ebr::queue::test::can_construct_queue ... ok
test ebr::queue::test::is_unique_receiver ... ok
    \end{lstlisting}
  \end{figure}

  We note that for this project a own benchmarking system was developed
  (Section~\ref{sec:profiling}), but we still show the official benchmark system here.
  Benchmarking is similar, except that we do not have a conditional compilation
  flag for benchmarks. For similarity, we can put benchmarks in the \code{bench}
  module. Benchmarks are annotated with \code{\#[bench]}. The benchmarks are ran
  with \code{cargo bench}, which runs all benchmark annotated functions.
  Listing~\ref{lst:bench-test} shows a sample benchmark, and
  Listing~\ref{lst:bench-test-output} shows sample output from a benchmark.
  Note that the benchmark system is not in stable Rust, so we need to use the
  nightly version. The code that is benchmarked is the closure passed to
  \code{test::Bencher::iter}.

  \begin{figure}[ht]
  \begin{lstlisting}[label=lst:bench-test,caption=An example benchmark in Rust]
mod bench {
    extern crate test;
    use super::Queue;
    #[bench]
    fn push(b: &mut test::Bencher) {
        let q = Queue::new();
        b.iter(|| {
            ::ebr::pin(|pin| {
                q.push(1, pin);
            });
        });
    }
}
    \end{lstlisting}
  \end{figure}
  \begin{figure}[ht]
    \begin{lstlisting}[label=lst:bench-test-output,caption=Sample output of
    \code{cargo bench},language=,numbers=none]
test ebr::bench::pin               ... bench:   18 ns/iter (+/- 0)
test ebr::queue::bench::push       ... bench:   57 ns/iter (+/- 12)
test hp::list::bench::remove_front ... bench:    2 ns/iter (+/- 0)
    \end{lstlisting}
  \end{figure}


  \clearpage,
  \section{Setup for cross-compilation}
  Other targets may be installed through \rustup{}. For instance, if we want to
  make \code{aarch64\-unknown-linux-gnu} available for cross-compilation we would run

  \begin{lstlisting}[language=Bash,numbers=none]
% rustup target add aarch64-unknown-linux-gnu
  \end{lstlisting}

  We can either pass in the target architecture to \cargo{} at each invocation,
  or we can configure \cargo{} to use another target by default. We show the latter.
  We make a new file in the project directory called \code{.cargo/config}
  containing Listing~\ref{lst:cargo/config}

  \begin{figure}[ht]
  \begin{lstlisting}[language=,
                     label=lst:cargo/config,
                     caption=Cargo configuration file for cross-compiling]
[build]
target = "aarch64-unknown-linux-gnu"

[target.aarch64-unknown-linux-gnu]
linker = "aarch64-linux-gnu-gcc"
  \end{lstlisting}
  \end{figure}
  This sets the default target to be \code{aarch-unknown-linux-gnu},
  and specifies the linked to be used.
  After this \code{cargo build} builds for the specified target.
  The executable can be found in \code{./target/aarch-unknown-linux-gnu/debug/}
  with the name of the project.


  \chapter{Benchmark Results\label{ch:benchmarks}}

\newcommand{\figuregrid}[1]{

  \rotatebox{90}{\hspace{-0.6cm}\footnotesize{\code{Queue::push}}}
  \foreach \scheme in {nothing, crossbeam, ebr, hp, hpspin} {%
    \begin{subfigure}{0.20\textwidth}
      \centering \footnotesize{\code{\scheme}}
      \includegraphics[width=\textwidth]{plots/#1-s:\scheme-b:queue_push}
    \end{subfigure}}

    \rotatebox{90}{\hspace{-0.6cm}\footnotesize{\code{Queue::pop}}}
  \foreach \scheme in {nothing, crossbeam, ebr, hp, hpspin} {%
    \begin{subfigure}{0.20\textwidth}
      \includegraphics[width=\textwidth]{plots/#1-s:\scheme-b:queue_pop}
    \end{subfigure}}

    \rotatebox{90}{\hspace{-0.9cm}\footnotesize{\code{Queue::transfer}}}
  \foreach \scheme in {nothing, crossbeam, ebr, hp, hpspin} {%
    \begin{subfigure}{0.20\textwidth}
      \includegraphics[width=\textwidth]{plots/#1-s:\scheme-b:queue_transfer}
    \end{subfigure}}

  \rotatebox{90}{\hspace{-0.6cm}\footnotesize{\code{List::remove}}}
    \begin{subfigure}{0.20\textwidth}
      \includegraphics[width=\textwidth]{plots/#1-s:nothing-b:list_remove}
    \end{subfigure}
  \makebox[0.20\textwidth]{}
  \foreach \scheme in {ebr, hp, hpspin} {%
    \begin{subfigure}{0.20\textwidth}
      \includegraphics[width=\textwidth]{plots/#1-s:\scheme-b:list_remove}
    \end{subfigure}}

    \rotatebox{90}{\hspace{-0.6cm}\footnotesize{\code{List::Real}}}
    \begin{subfigure}{0.20\textwidth}
      \includegraphics[width=\textwidth]{plots/#1-s:nothing-b:list_real}
    \end{subfigure}
  \makebox[0.20\textwidth]{}
  \foreach \scheme in {ebr, hp, hpspin} {%
    \begin{subfigure}{0.20\textwidth}
      \includegraphics[width=\textwidth]{plots/#1-s:\scheme-b:list_real}
    \end{subfigure}}
}

All measured timings for all CPU's we have ran the benchmarks on are listed in
this chapter. The timings are plotted in box plots, so that outliers are easy to
spot. This is especially relevant for our use case, since some of the
reclamation schemes only reclaim memory occasionally.  Note that the range on
the y axis varies. The box plots are organized in 5x5 grids, where each column
is a reclamation scheme and each row is a benchmark. Thus, we read left-to-right
to compare the different schemes.

We note again that Crossbeam does not offer a List implementation, and is such
excluded from the benchmarks that is using a List.


\clearpage
\section{Intel\textregistered{} i7--7500U}
\begin{figure}[ht]
  \figuregrid{laptop}
\end{figure}

\clearpage
\section{Intel\textregistered{} i7--4770}
\begin{figure}[ht]
  \figuregrid{gribb}
\end{figure}

\clearpage
\section{Intel\textregistered{} Xeon\textregistered{} E5--2620}
\begin{figure}[ht]
  \figuregrid{server}
\end{figure}

\clearpage
\section{Ryzen 7 1700}
\begin{figure}[ht]
  \figuregrid{ryzen}
\end{figure}

\clearpage
\section{Cavium ThunderX}
\begin{figure}[ht]
  \figuregrid{scaleway}
\end{figure}



\end{appendices}

\end{document}
